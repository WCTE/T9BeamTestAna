%%%%%%%%%%%%%%%%%%%%%%%%%%


\documentclass{article}

% \usepackage[utf8]{inputenc}
% \usepackage[latin2,utf8]{inputenc}


\newcommand{\ttof}{\ensuremath{t_\mathrm{TOF}}}

\usepackage{graphicx}

%%%%%%%%%%%%%%%%%%%%%%%%%%
%%%%%%%%%%%%%%%%%%%%%%%%%%
%%%%%%%%%%%%%%%%%%%%%%%%%%

\begin{document}


%%%%%%%%%%%%%%%%%%%%%%%%%%
\section{ToF calculation}

\subsection{Tof and beam momentum}

%\includegraphics[width=0.33\textwidth]{} &

$$ p = mc \frac{\beta}{\left[  1 - \beta^2 \right]^{1/2}}$$
$$ \beta = \left[  1 + \left( \frac{mc}{p} \right)^2 \right]^{-1/2}$$


Momentum measurement using the time of flight (TOF) of muons or pions $\ttof^{\mu/\pi}$ can be performed by measuring TOF across the finite distance between the TOF scintillators $L=349\,\mathrm{cm}$ using the peak arrival times.

$$ t_\mathrm{TOF} = \frac{L}{c} \left[ 1 + \left( \frac{mc}{p}  \right)^2   \right]^{1/2}$$

\begin{eqnarray}
  \Delta t_\mathrm{TOF}^{\mu/\pi} &\equiv&  t^{\mu/\pi}_\mathrm{TOF} - t^e_\mathrm{TOF} \\
    &=& \frac{L}{c}  \left\{  \left[ 1 + \left( \frac{m_{\mu/\pi} c}{p}  \right)^2   \right]^{1/2} -  \left[ 1 + \left( \frac{m_e c}{p}  \right)^2   \right]^{1/2}   \right\} \\
     &\approx& \frac{L}{c}  \left\{  \left[ 1 + \left( \frac{m_{\mu/\pi} c}{p}  \right)^2   \right]^{1/2} - 1  \right\} 
\end{eqnarray}

    Given the beam momenta provided by the setup, electrons can be safely regarded as ultrarelativistic particles, with $1-\beta < 10^{-5}$.
From the measured fitted TOF mean of muons or pions $\Delta\ttof^{\mu/\pi}$ subtracted by the TOF observed in the electron peak the mean measured momentum of muons or pions 
can be inferred as 
$$p_{\mu/\pi}^\mathrm{meas} = m_{\mu/\pi} \, c \left[ \left( \frac{c}{L} \Delta\ttof^{\mu/\pi} + 1 \right)^2  - 1\right]^{-1/2}\,.$$
The measured momentum bias $\Delta p^\mathrm{bias}_{\mu/\pi}$ w.r.t. the TOF of electrons is then the difference between the extracted mean momentum, for each particle type, and of the nominal beam momentum reported by the T9 setup
$\Delta p^\mathrm{bias}_{\mu/\pi} =  p_{\mu/\pi}^\mathrm{meas} -  p^\mathrm{T9}$. 

Uncertainties propagation:
$$ \sigma_p = \frac{ \left( p_{\mu/\pi}^\mathrm{meas} \right)^3}{ \left(  m_{\mu/\pi} \, c \right)^2} \left( \frac{c}{L} \,\Delta\ttof^{\mu/\pi} + 1  \right)  \frac{c}{L} \sqrt{\sigma^2_{t,\mu/\pi} + \sigma^2_{t,e}}$$

N.b.
$$ \frac{L}{c} = \left( \frac{c}{L} \right)^{-1} \doteq 11.64\,\mathrm{ns}$$
$$ L = 3.49\, \mathrm{m}\,, \quad c = 299792458\, \mathrm{m}\,\mathrm{s}^{-1}$$

\clearpage


$$ f(\beta|A,B,C) = \frac{A}{\beta^2} \left( \log \frac{B\beta}{\sqrt{1-\beta^2}} - \beta^2 \right) + C $$

$$ f(\beta|A,B) = \frac{A}{\beta^2} \left( \log \frac{B\beta}{\sqrt{1-\beta^2}} - \beta^2 \right)$$

$$ \vec{r} = (x,y) = \frac{ \sum\limits_{i=0}^3  \mathcal{C}_i \, \vec{w}_i }{\sum\limits_{i=0}^3  \mathcal{C}_i}$$

$$ x = \frac{ -\frac{x_D}{2} \mathcal{C}_L + \frac{x_D}{2} \mathcal{C}_R }{\mathcal{C}_L + \mathcal{C}_R}$$



\end{document}

%%%%%%%%%%%%%%%%%%%%%%%%%%
%%%%%%%%%%%%%%%%%%%%%%%%%%
%%%%%%%%%%%%%%%%%%%%%%%%%%
